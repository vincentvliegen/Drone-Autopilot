\subsubsection{Afremmen}
\noindent {\em Auteur: Matthias Van der Heyden}
\\
\\
De drone is zodanig ontworpen dat hij tot stilstand kan komen in de co\"ordinaten van het opgegeven target. Dit doet hij door voortdurend de targetpositie en de maximale toelaatbare acceleratie mee te geven. Deze acceleratie wordt berekend aan de hand van de drone zijn huidige snelheid en afstand tot de targetpositie.
\\
\\
Afhankelijk van zijn snelheid ten opzichte van de afstand tot het doel moet de drone afremmen (bij $\frac{snelheid}{afstand}  \geq empirische factor$) of versnellen (bij $\frac{snelheid}{afstand}  < empirische factor$). De grootte van de acceleratie of deceleratie is steeds de maximaal mogelijke waarde, rekening houdend met de huidige wind, tenzij de afstand tot het doel kleiner is dan een bepaalde afstand. In dat geval verkleint de waarde met een afstandsafhankelijke factor.