\noindent {\em Auteurs: Arne Vlietinck; redactie: Bram Vandendriessche, Laura Vranken \& Arne Vlietinck}
\\\\
Dit verslag behandelt het ontwerp en de implementatie van een Autopilot en Virtual Testbed voor een drone. In de Simulator wordt er in komende mijlpalen verwacht met polyhedra te werken in plaats van bollen. Voor de tweede mijlpaal wordt het wire protocol voorzien. In mijlpaal drie scant de drone het object en geeft cre\"eert het een grafische output ervan. De laatste mijlpaal is een uitgebreidere scanfunctie van de volledige wereld. 
\\
De Autopilot zorgt voor een correcte aansturing van de drone. Hij bepaalt de vliegroute op basis van informatie verkregen van twee camera's die zich op de drone bevinden. Ook de variabelen verkregen via de connectie tussen Simulator en Autopilot zijn van belang. De Autopilot leidt daaruit de beweging van de drone af en laat de simulator deze uitvoeren.
\\
Het programma gebruikt \textit{GUI's} als interactief medium. Deze geven de mogelijkheid het camerastandpunt te kiezen. Ook de voltooiingsgraad, snelheid en positie van de drone worden weergegeven.