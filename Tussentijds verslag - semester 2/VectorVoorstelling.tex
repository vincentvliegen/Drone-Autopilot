\subsubsection{Ruimtelijke voorstelling d.m.v. vectoren}
\noindent {\em Auteur: Vincent Vliegen}
\\
\\
De autopilot krijgt via de vernieuwde interface een positie en een volledige ori\"entatie mee. Deze maken het mogelijk voor de autopilot om accuraat verplaatsingen en rotaties in de ruimte te bechrijven. De extra gegevens zorgen ervoor dat de beweging van de drone, de krachten in de ruimte en andere variabelen als vectoren kunnen beschreven worden. Ook is er voldoende informatie aanwezig om een goede benadering te maken van de effecten die de onbekende windkrachten veroorzaken.
\\
\\
De voorstelling met vectoren voorziet de mogelijkheid om het vliegtraject van de drone te bepalen en bij te sturen naar wens. De positie van de drone bepaalt enkele cruciale gegevens: de snelheid, de verplaatsingsrichting en een onverwachte translatie door de wind. De ori\"entatie daarentegen is nodig om de nodige rotaties te berekenen om de vliegrichting te corrigeren, maar ook om de in- en uitwendige krachten op de drone juist te ori\"enteren en om de eventuele invloed van windrotaties in te perken. Met al deze gegevens is de autopilot in staat om de gewenste thrust en rotatiesnelheden te bepalen, om op een zo effici\"ent mogelijke manier naar een gegeven targetpositie te vliegen.
\\
\\
Daarnaast worden objecten nu ook voorgesteld met vectoren. Aan de hand van de camerabeelden wordt de positie van het object relatief tegenover de drone bepaald. Wanneer de ori\"entatie en positie van de drone in rekening worden gebracht, is ook de positie van het object in de ruimte bekend. Deze kan dan gebruikt worden als targetpositie of voor andere doeleinden.