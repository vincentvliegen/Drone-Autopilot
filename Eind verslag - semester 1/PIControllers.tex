\noindent
Om een PI controller te kunnen gebruiken, is het belangrijk om te weten wat het is en wat het doet. Volgens \cite{website:PIDController} fungeert de controller als een controlelus-feedback mechanisme. Dit systeem gaat proberen de fout op de meetwaarde proberen te corrigeren. Het berekent namelijk continu de foutwaarde e(t) als het verschil tussen de verlangde waarde en de gemeten waarde. Op basis van deze gegevens berekent de controller de nodige correctie.
\\
\\
Er is gekozen voor een PI controller i.p.v. een PID controller, aangezien die ook in de meeste re\"ele systemen wordt gebruikt. Bovendien is de term D (Derivative) heel gevoelig aan ruis en zou die een bijkomende moeilijkheid vormen om af te stellen.
\\
PI staat voor Proportional-Integral. Het P-deel zorgt ervoor dat het verschil in verlangde waarde en de gemeten waarde met een factor $K_p$ wordt versterkt. 
I zorgt voor een constante sommatie van de fout en vergroot de correctiewaarde afhankelijk van hoelang er een fout bestaat tussen gemeten en verlangde waarde. Dit wordt met een factor $K_i$ versterkt. Deze constanten worden manueel bepaald door de reactie van de controller uit te plotten in grafieken (MSS FOTO?) en hun gedrag te optimaliseren door deze co\"effici\"enten aan te passen.
\\
\\
Tenslotte wordt de correctie dan op basis van de volgende formule berekend.
\begin{equation}
	u(t) = K_p * e(t) + K_i * \int e(t) dt
\end{equation}
Hier is u(t) de correctie, $K_p$ de proportional constante, $K_i$ de integral constante en e(t) de foutwaarde op tijdstip t.
\\
\\
Nu gaan we deze theorie toepassen op ons eigen project.
\\
Ten eerste bekijken we de roll controller. Hij heeft als doel de roll gelijk aan nul te houden. Dit is dan ook ineens de verlangde waarde. De controller begint slechts bij te sturen wanneer de gemeten roll waarde groter is dan 0,2\degree in positieve of negatieve zin. Bijsturen gebeurt door de correctiewaarde als input te nemen voor de roll rate.
\\
Vervolgens hebben we de yaw controller. Deze werkt enkel wanneer de doel bol in beeld is, aangezien hij een ori\"entatiepunt nodig heeft om de yaw te kunnen bepalen. De verlangde waarde is gelijk aan een horizontale afwijking $\alpha$ van nul graden. De controller gaat dit bijsturen op dezelfde manier als bij de roll controller.
\\
De hoogte controller stelt de thrust waarde in en verschilt licht van de anderen doordat hij de correctiewaarde optelt bij de standaard waarde om de zwaartekracht tegen te gaan. Verder probeert hij de verticale afwijking naar nul te brengen.
\\
De pitch controller wordt gebruikt om zo snel mogelijk te kunnen gaan hoveren en dus een pitch van nul te krijgen. Wederom op dezelfde manier.
\\
Tenslotte is er ook nog een afstandscontroller. Deze gaat proberen de afstand naar nul te reduceren met stappen van 0.1m en helpt bij het voorwaarts vliegen om de pitchrate in te stellen, zoals eerder uitgelegd.