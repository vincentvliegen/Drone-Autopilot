\noindent
Wanneer de Autopilot iets in beeld gekregen heeft, zal hij starten met zijn positie relatief tegenover zijn doel te bepalen. Dit gebeurt in volgende stappen.
\\
Ten eerst wordt de diepte bepaald. Dit kan met behulp van de formule van stereo vision \cite{website:techbriefs} uitgewerkt worden.
Zie Figuur \ref{fig:DiepteberekeningDroneEnDoel} voor een grafische weergave van de berekening. Z stelt de diepte [m] voor, c de afstand [m] tussen de camera's, f de focale afstand [pixel] en $x_1$ en $x_2$ stellen de afstand [pixel] voor tussen het middelpunt van het beeld en het middelpunt van de bol op het beeld voor.
\begin{figure}[h]
	\centering
	\includegraphics[width=0.3\textwidth]{DiepteberekeningDroneEnDoel.png}
	\caption{Diepteberekening tussen drone en doel. In formulevorm: \(Z = \frac{c * f}{x_1 - x_2}\).}
	\label{fig:DiepteberekeningDroneEnDoel}
\end{figure}
\\
Vervolgens bepalen we de fout op de horizontale hoek, $\alpha$. Hiermee wordt de afwijking tussen de horizontale positie van de bol en het midden van de drone bedoeld. Deze formule wordt afgeleid via de goniometrische regels. Zie Figuur \ref{fig:RelatieveHorizontaleHoek} voor grafische ondersteuning. De hoek $\delta$ stelt hier de helft van de horizontale hoek die het beeld overspant, voor en b stelt de helft van de breedte van het beeld voor. Onder de figuur, kan eveneens ook de uitwerking van de formule gevonden worden.
\\
Tenslotte bepalen we ook de fout op de verticale hoek, $\beta$. Dit is de afwijking van de hoogte van de bol ten opzichte van de hoogte van de drone. Wederom afgeleid via de goniometrie en weergegeven in Figuur \ref{fig:RelatieveVerticaleHoek}, met $y_2$ de verticale afstand [pixel] tussen de het middelpunt van het beeld en het middelpunt van de bol.
\begin{figure}[h]
	\centering
	\begin{minipage}{.45\textwidth}
		\centering
		\includegraphics[width=0.8\textwidth]{RelatieveHorizontaleHoek.png}
		\caption{Relatieve horizontale hoek.}
		\label{fig:RelatieveHorizontaleHoek}
	\end{minipage}
	\begin{minipage}{.45\textwidth}
		\centering
		\includegraphics[width=0.96\textwidth]{RelatieveVerticaleHoek.png}
		\caption{Relatieve verticale hoek.}
		\label{fig:RelatieveVerticaleHoek}
	\end{minipage}%
	\caption*{In Figuur \ref{fig:RelatieveHorizontaleHoek} wordt de relatieve horizontale hoek tussen drone en doel weergegeven door \(\tan(\alpha) = \frac{x-\frac{c}{2}}{Z}\), voor de volledige uitwerking zie formule \ref{eq:RelatieveVerticaleHoekBegin} tot \ref{eq:RelatieveVerticaleHoekEind}.\\
		In Figuur \ref{fig:RelatieveVerticaleHoek} wordt de relatieve verticale hoek weergegeven tussen drone en doel. De relatie wordt gegeven door volgende formule: \(\tan(\beta) = \frac{y_1}{f}\).}
\end{figure}
\begin{figure}[h]
	\centering
	\begin{minipage}{.5\textwidth}
		Berekening brandpuntsafstand f:
		\begin{equation} \label{eq:RelatieveVerticaleHoekBegin}
		\tan(\delta) = \frac{b}{f}
		\end{equation}
		\begin{equation} 
		f = \frac{b}{\tan(\delta)}
		\end{equation}
	\end{minipage}
	\begin{minipage}{.45\textwidth}
		Berekening afstand x:
		\begin{equation} 
		\frac{x_2}{f} = \frac{x}{z}
		\end{equation}
		\begin{equation} \label{eq:RelatieveVerticaleHoekEind}
		x = z * \frac{x_2}{f}	
		\end{equation}
	\end{minipage}%
\end{figure}
\\
Nadat de drone zijn positie bepaald heeft, kan hij zijn fouten (horizontale en verticale hoek en eventuele roll) bijsturen a.d.h.v. PI controllers die beslissen over respectievelijk yaw, thrust en roll bewegingen. Meer info over de werking van de controllers wordt gegeven in subsectie \ref{subsec: PI Controllers}.
\\
\\
Nu rest er ons enkel nog de pitch te bepalen zodat we onder een ingestelde thrust voorwaarts richting de bol kunnen bewegen. Het eerste idee was om dit op basis van een soort van snelheidscontroller te doen, aangezien de snelheid afhankelijk is van de pitch en dat we hierdoor ook de snelheid konden controlleren en laag houden. De snelheid is echter niet te bepalen door de onbekende windkrachten of door de numeriek wiskundige beperkingen op afgeleiden berekenen. Bijgevolg zijn we overgegaan op het tweede plan. Dit plan houdt in dat de drone pitcht met een rate op basis van een afstandscontroller, zodat hij een kleine afstand overbrugt, terug recht komt wat de snelheid een klein beetje afbouwt en dan weer opnieuw pitcht om verder te gaan. Wanneer er tegenwind is, zal de afstandscontroller de drone toelaten om verder te pitchen om toch op zijn doel te raken.
\\
\\
Tenslotte moet dit proces herhaaldelijk worden uitgevoerd ten gevolge van de invloed van wind. De wind kan de drone namelijk uit koers brengen. Hierdoor zal de drone telkens zijn positie moeten herberekenen en zich opnieuw ori\"enteren en de controllers laten bijsturen.
\\
De bijkomstige moeilijkheid van obstakels zal verder worden uitgewerkt in subsectie \ref{subsec: Obstakels}.