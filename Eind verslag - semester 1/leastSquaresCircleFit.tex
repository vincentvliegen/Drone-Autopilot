\noindent
De nauwkeurigheid van de acties van de autopilot hangt af van de precisie van de informatie die de \textit{ImageCalculations} terug geeft. De afstand en positionering van alle objecten rond de drone zijn afhankelijk van hun centrum, waarvan de co\"ordinaten op regelmatige basis worden herberekend. Om de meest exacte posities van deze objecten te verkrijgen, maakt de \textit{ImageCalculations} gebruik van en zwaartepuntsberekeningen en een kleinste kwadraten circle fit algoritme. Aangezien het circle fit algoritme beter werkt bij niet volledig zichtbare bollen, zal de \textit{ImageCalculations} altijd hiervoor kiezen, tenzij er niet voldoende pixels zitten op de cirkelomtrek.
\\
Het kleinste kwadraten circle fit algoritme probeert een cirkel te vinden die zo goed mogelijk overeenkomt met de afbeelding van een gegeven bol\cite{website:kleinsteKwadraten}. Het gebruikt enkel de pixels op de omtrek van de cirkel. Het kleinste kwadraten algoritme berekent voor elk van deze pixels de horizontale en verticale afwijking tegenover de gemiddelde positie van deze pixels in het \((x,y)\)-assenstelsel. Deze afwijkingen worden dan omgezet naar co\"ordinaten in een \((u,v)\)-assenstelsel, waar de fout bepaald kan worden. Het kleinste kwadraten algoritme berekent dan de \((u,v)\)-co\"ordinaten van het middelpunt waarvoor de som van de gekwadrateerde fouten minimaal is. Tot slot wordt er bij deze co\"ordinaten de eerder berekende gemiddeldes opgeteld wat resulteert in het middelpunt van de best passende cirkel in het \((x,y)\)-assenstelsel. 