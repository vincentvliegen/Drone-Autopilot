% Geef de resultaten op compacte en heldere manier weer, bv. met tabellen of grafieken. Denk na over de beste manier om de resultaten weer te geven. Ze moeten je conclusies ondersteunen. Formuleer de conclusies.
\noindent {\em Auteur: Vincent Vliegen}
\\
\\
De Autopilot is voorzien van enkele \textit{JUnit}-testklassen. Deze testen verschillende methodes op hun nauwkeurigheid en correctheid. Zo is het eenvoudig de juistheid van deze veelgebruikte methodes na te gaan.
\\
\\
De \textit{ImageCalculationsTest} is uitgerust met testen voor elke methode in de \textit{ImageCalculations}-klasse. Er is gebruik gemaakt van een anonieme klasse die de \textit{Camera} interface implementeert, zodat er afbeeldingen naar keuze gegenereerd kunnen worden. Deze afbeeldingen zijn omwille van hun grote hoeveelheid informatie beperkt tot een minimale grootte van \(9\text{x}9\) en \(10\text{x}10\) pixels, dit vereenvoudigt immers het testen. Desalniettemin zijn er ook twee bitmap bestanden ter beschikking met elk een representatief beeld van een rode bol. Dit geeft een voldoende nauwkeurige, cirkelvormige afbeelding van een rode bol voor de berekening van desbetreffend middelpunt. De testen tonen aan dat wanneer het centrum van de bol zich buiten beeld bevindt, de benadering van het middelpunt beter is bij berekeningen via het least square circle fit algoritme dan bij berekeningen van het zwaartepunt van de zichtbare pixels.
\\
\\
De \textit{PhysicsCalculationsTest} verschaft testmethodes voor de \textit{PhysicsCalculations}-klasse. Ook hier zijn anonieme klassen ge\"implementeerd, namelijk voor de \textit{Camera} en \textit{Drone} interfaces. Deze testen geven weer dat de gebruikte formules in iedere situatie voldoende zijn om alle fysische data nauwkeurig te berekenen.
