\noindent
Ten eerste moeten de beelden die de Autopilot van de Virtual Testbed binnenkrijgt, geanalyseerd worden. Dit gebeurt door iteratief de kleurwaarden van elke pixel te vergelijken met de waarde van de opgegeven kleur, indien er al een doel beslist is. Anders zullen de pixels gegroepeerd worden per kleur dat voorkomt in een HashMap. De gekleurde pixels worden bijgehouden door hun positie ten opzichte van het beeld, uitgedrukt in rij en kolom, op te slaan. We baseren onze berekeningen op het midden van de bol. Dit kan benaderd worden op twee manieren: via het zwaartepunt of de kleinste-kwadratenmethode op de randpunten van de cirkel. Het zwaartepunt van een bepaalde kleur pixels is te berekenen via het gemiddelde van de opgeslagen co\"ordinaten. De kleinste-kwadratenmethode zoekt daarentegen eerst de randpunten uit van de cirkel. Deze worden vervolgens gebruikt in een algoritme (SUBSECTIEVERWIJZING), dat de cirkel bepaalt die het beste past in de gegeven randpunten. Hieruit kan dan de positie van het centrum van de bol bepaald worden. \cite{website:kleinsteKwadraten} De Autopilot zal eerst gebruik maken van de kleinste-kwadratenmethode en overschakelen op de zwaartepuntberekening wanneer er onvoldoende randpunten zijn, aangezien deze minder nauwkeurig is wanneer het middelpunt buiten beeld ligt.
TODO: Vincent moet bij algoritmen zijn kleinste kwadraten meer uitleggen!
\\
Indien de Autopilot geen gekleurde pixels detecteert, zal de drone geleidelijkaan de wereld afscannen. Hierover meer info in subsectie \ref{subsec: Scannen wereld}.