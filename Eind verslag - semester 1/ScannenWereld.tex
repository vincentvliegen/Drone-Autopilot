\noindent
Wanneer de drone geen bollen meer in beeld heeft, moet hij rond zich beginnen zoeken totdat hij een nieuwe bol waarneemt. Om de hele wereld af te scannen, voeren we volgende stappen uit.
\\
\\
Ten eerste begint de drone rond zijn as te draaien d.m.v. een yaw beweging. Hij krijgt de opdracht dit te doen gedurende 540\degree. We nemen een overschot van 180\degree  zodat hij zeker volledig rondgedraaid heeft, mits invloed van eventuele rotationele wind.
\\
Vervolgens vliegt de drone achteruit over een bepaalde afstand zodat mogelijke bollen die zich boven of onder de drone bevinden ook zichtbaar worden. Dit gebeurt door eerst omhoog te pitchen en achterwaartse snelheid op te bouwen, om vervolgens weer horizontaal te komen zodat ook de bollen onderaan zichtbaar blijven. We maken van de opgebouwde snelheid in het begin gebruik om ver genoeg achteruit te drijven.
\\
\\
Indien we na dit proces nog steeds niks waargenomen hebben, herstarten we het van het begin. 