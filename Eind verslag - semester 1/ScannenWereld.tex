\noindent
Wanneer de drone geen bollen meer in beeld heeft, moet hij rond zich beginnen zoeken totdat hij een nieuwe bol waarneemt. Om de hele wereld te scannen, wordt volgende strategie toegepast.
\\
\\
Ten eerste begint de drone rond zijn as te draaien d.m.v. een yaw beweging. Hij krijgt de opdracht dit te doen gedurende \(630\degree\). Een overschot van \(270\degree\) wordt in rekening gebracht, zodat hij zeker volledig rondgedraaid heeft. Dit gebeurt om eventuele rotationele invloeden van de wind op te heffen.
\\
Vervolgens vliegt de drone achteruit over een bepaalde afstand zodat mogelijke bollen die zich boven of onder de drone bevinden ook zichtbaar worden. Dit gebeurt door eerst omhoog te pitchen en achterwaartse snelheid op te bouwen, om vervolgens weer horizontaal te komen zodat ook de bollen onderaan zichtbaar blijven. We maken van de opgebouwde snelheid in het begin gebruik om ver genoeg achteruit te vliegen.
\\
\\
Indien er na dit proces nog steeds niks waargenomen is, wordt het geheel opnieuw aangeroepen. 