\noindent {\em Auteur: Matthias Van der Heyden} 
\\
\\
De \textit{GUI} van de Autopilot heeft twee functies: de gebruiker de mogelijk geven een opdracht voor de drone te selecteren en de voortgang van het bereiken van de volgende bol weergeven.
\\
\\
Voor het selecteren van een opdracht is er een dropdownmenu voorzien dat gebruik maakt van \textit{JComboBox} uit de \textit{Swing library}\footnote{\textit{Swing library}: https://docs.oracle.com/javase/7/docs/api/javax/swing/package-summary.html} van Java. Wanneer de gebruiker een optie aanduidt, verandert de boolean van deze optie naar \textit{true}. Dit zorgt ervoor dat de juiste commando's voor deze opdracht uitgevoerd worden. Er zijn twee mogelijke opdrachten: \textit{Fly shortest path}, die de drone de kortste weg door de verschillende bollen laat vliegen, en \textit{Fly to red orb}, die er voor zorgt dat de drone een rode bol zoekt en er heen vliegt. Deze laatste optie kan eenvoudig uitgebreid worden naar bijvoorbeeld \textit{Fly to colored orb} waarbij de gebruiker in een pop-up de gewenste kleur invult of aanduidt op een kleurenpalet. 
\\
\\
Een \textit{JProgressBar} uit de \textit{Swing library} van Java geeft in de \textit{GUI} de voortgang van de drone tot de volgende bol weer. Ze krijgt de kleur van de bol waar  de drone naartoe vliegt. Bij elke berekening van de afstand tot het doel wordt deze ge\"{u}pdatet. Is de afstand groter dan de laatste grootste afstand tot het doel, dan stelt de progress bar deze nieuwe afstand in als het nieuwe maximum en is de voltooiingsgraad weer nul. Is de afstand kleiner, dan is de nieuwe voltooiingsgraad gelijk aan 100 procent verminderd met de verhouding van de grootste afstand en de huidige afstand.  
