{\em Auteur: Bram Vandendriessche}\\

\noindent
Het bouwen van het Testbed is begonnen met de belangrijke en moeilijke keuze voor de te gebruiken 3D-omgeving. Deze omgeving moest vlot onder de knie te krijgen zijn vanwege de snelle deadlines en ook flexibiliteit was een vereiste. Blender\footnote{\url{https://www.blender.org/}}, JMonkeyEngine\footnote{\url{http://jmonkeyengine.org/}} en OpenGL\footnote{\url{https://www.opengl.org/}} haalden als meest geschikte kandidaten de laatste ronde. Andere keuzes, zoals Unity\footnote{\url{https://unity3d.com/}}, vielen af door bijvoorbeeld het gebrek aan kennis van \texttt{C\#} en aan tijd om hier verandering in te brengen.\\
~\\
Blender is een erg uitgebreid programma met tal van mogelijkheden om 3D-objecten en -werelden te maken en te manipuleren. Het maakt gebruik van Python, een programmeertaal met een vrij eenvoudige leercurve voor wie al programmeerervaring heeft. Daarentegen is Blender uitdagend en tijdrovend om te beheersen, net door alle mogeljke functies. Aangezien we bovendien gepland hadden in Java te werken, moest gezocht worden naar een manier om Java en Python te verbinden. Blender zou dan vanuit Java gestart moeten worden, wat een omslachtig proces leek. Vooral omwille van de extra leertijd en omweg, werd niet voor Blender gekozen. \\
De tweede optie, JMonkeyEngine, leek erg gebruiksvriendelijk, had een goede tutorial en enkele handige ingebouwde functies, zoals het vastpinnen van een camera op een object (wat erg nuttig is voor het project). Anderzijds leek de community-ondersteuning (op fora als StackOverflow) bij specifieke zoektermen die bij het testen geprobeerd werden, erg beperkt. Hierdoor viel ook deze optie weg ten opzichte van OpenGL.\\
\\
De keuze viel uiteindelijk op OpenGL. Ondanks een erg moeizaam begin, is deze toch de goede gebleken. OpenGL kan rechtstreeks in Java worden gebruikt, heeft een brede community met heel wat tutorials en werkt erg flexibel.






%TODO eventueel nog zeggen dat drone weergegeven wordt door balk, kan later nog aangepast worden

%TODO Andere ontwerpkeuzes?