\noindent
{\em Auteur: Arne Vlietinck}\\
\\
Dit verslag behandelt het ontwerp en de implementatie van een Autopilot en Virtual Testbed voor een drone. In de simulator is voor de eerste mijlpaal een rode bol in een witte ruimte te zien. Bij de tweede mijlpaal wordt deze wereld uitgebreid met een windkracht in een willekeurige richting. Voor de derde mijlpaal zijn er verschillende bollen zichtbaar die allemaal op een zo efficiënt mogelijke manier moeten worden doorprikt. Bij de laatste mijlpaal zijn niet alle bollen direct zichtbaar. Een uitbreiding is het ontwijken van grijze obstakels.
\\
De Autopilot zorgt voor een correcte aansturing van de drone. Hij bepaalt de vliegroute op basis van informatie verkregen van twee camera's die zich op de drone bevinden. Hierbij is de relatieve plaatsbepaling van de bol ten opzichte van de drone van belang. De Autopilot leidt daaruit de beweging van de drone af en laat de simulator deze uitvoeren.
\\
Het programma wordt interactief gemaakt door het gebruik van \textit{GUI's}. Deze geven de mogelijkheid het camerastandpunt te kiezen, extra bollen toe te voegen, ook de voltooiingsgraad, snelheid en positie van de drone worden weergegeven. Daarnaast kunnen externe factoren (wind en gravitatieconstante) aangepast worden.