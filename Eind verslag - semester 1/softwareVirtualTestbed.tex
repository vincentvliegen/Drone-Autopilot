{\em Auteur: Bram Vandendriessche}
\\

\noindent
Voor de simulator is geprobeerd om zo modulair mogelijk te werken, zodat nieuwe opstellingen of constraints gemakkelijk kunnen worden toegevoegd zonder dat dit voor problemen zou zorgen in de bestaande structuur. De klasse \textit{World} staat centraal in de simulator. Zij bevat alle basiselementen voor zowel het fysische als voor het 3D-gedeelte van de wereld. Het fysische aspect wordt behandeld door de klasse \textit{Physics}, wat al eerder werd besproken in sectie \ref{sec:AlgoritmesVirtualTestbed}. 
\\

\noindent
\textit{World} is ge\"implementeerd als een subklasse van \textit{GLCanvas} en legt de basis voor de 3D-weergave van de verschillende opstellingen. 
De klasse houdt bij welke \textit{WorldObjects} ze bevat (\textit{Spheres}, \textit{SimulationDrones}, \textit{ObstacleSpheres}) en implementeert enkele belangrijke functies voor de 3D-visualisatie ervan. De initialisatie van de wereld gebeurt door \texttt{init()}. De functie \texttt{draw()} roept de \texttt{draw()}-functie op van elk \textit{WorldObject} dat de wereld bevat. De \texttt{display()}-functie maakt meerdere keren gebruik van \texttt{draw()} om de wereld zowel naar het venster van de \textit{GUI} te renderen als naar de \textit{framebuffer objects} die dienen voor het offscreen renderen (voor \texttt{takeimage()}).\\
Bovendien bevat \textit{World} ook een abstracte methode \texttt{setup()}. Deze kan naar wens ge\"implementeerd worden door elke subklasse, zodat per mijlpaal een specifieke opstelling kan worden vastgelegd. Voor Milestone 1.1 is dit bijvoorbeeld enkel een rode bol, een drone en enkele camera's. Voor een wereld die vanuit een invoerbestand wordt opgebouwd met behulp van de parser, bestaat \textit{WorldParser}, waarbij de \texttt{setup()}-functie gebruik maakt van de geparste waarden.\\
\\
Uiteraard is er niets aan een 3D-wereld indien hij niet bekeken kan worden. De klasse \textit{GeneralCamera} werd ingevoerd om te kunnen werken met het concept van een camera, die een positie en kijkrichting krijgt. Zulke camera's kunnen dan op verschillende punten geplaatst worden (vastgelegd in de \texttt{setup()} van de wereld), zodat de gebruiker de voortgang van de drone goed kan volgen. Een uitbreiding op dit concept is de \textit{DroneCamera}, die zich voor zijn positie en ori\"entatie op de drone (waartoe hij behoort) zal baseren. Dit laatste type kan gebruikt worden voor de ``ogen'' van de drone en deze klasse implementeert de \textit{Camera}-interface van de \textit{API} voor de verbinding van Autopilot en Testbed.\\
Deze rol vult \textit{SimulationDrone} in voor de \textit{Drone}-interface. Deze klasse tekent de drone als een blauwe balk.

