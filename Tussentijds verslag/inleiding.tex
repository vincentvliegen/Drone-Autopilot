{\em Auteur: Laura Vranken \& Arne Vlietinck}\\

% \emph{[In de inleiding schets je de context, probleemstelling en doelstellingen van het project.  Je kunt ook kort aangeven wat er wel en niet in het verslag staat (bij een tussentijds verslag).]} 
\noindent
Drones zijn de laatste jaren enorm in populariteit toegenomen en blijven hierdoor ook in positieve zin evolueren. Ze worden tegenwoordig gebruikt voor talloze toepassingen. De bekendste toepassing bevindt zich binnen Defensie, die drones gebruiken om informatie te verkrijgen over vijandelijk gebied zonder mensenlevens te moeten riskeren. Daarnaast hebben ook grote bedrijven (o.a. Amazon\footnote{Amazon Prime Air}) de weg naar deze technologie gevonden. De toekomst brengt echter nog veel meer voordelen. Enkele voorbeelden zijn veiligheidsinspectie van windturbines of elektriciteitslijnen, luchtsteun bij zoek- en reddingsoperaties, bewaking, luchtfotografie enzovoort. (\cite{website:microdrones})
\\
Wanneer een drone autonoom functioneert, is een betrouwbare aansturing door de Autopilot van levensbelang. Hij moet namelijk bestand zijn tegen alle (eventueel extreme) weersomstandigheden.
\\
\\
In dit verslag gaan we verder in op hoe de autonome aansturing van een drone (en meer bepaald een quadcopter) gebeurt. Er wordt uitgegaan van een drone waarop twee camera's bevestigd zijn op een zekere afstand van elkaar. Beiden zijn ze voorwaarts gericht. Op basis van deze beelden moeten afstand en positie van het doel ingeschat worden en nieuwe bewegingsopdrachten voor de drone gegenereerd worden. Aangezien er geen hardware ter beschikking was, moet er een Virtual Testbed ontworpen worden. Dit is een softwaresysteem dat een fysieke opstelling van een drone en camera's simuleert. \cite{arcticle:opgavePeno} De simulator genereert beelden van de drone in verschillende standpunten a.d.h.v. de verkregen bewegingsopdrachten van de Autopilot. 
\\
\\
De Autopilot en Virtual Testbed moeten zo ontworpen worden dat de drone in staat is om zijn doel, een rode bol, te lokaliseren en er naar toe te vliegen. Dit eventueel onder lichte invloed van wind in willekeurige richtingen. Bovendien moet ook voor beiden een \textit{GUI} ontworpen worden die ter beschikking staat van de gebruiker. De \textit{GUI} toont de vooruitgang en laat de gebruiker toe allerlei informatie (snelheid, positie en verschillende camerastandpunten) op te vragen.
\\
\\
De tekst is als volgt opgebouwd. Eerst wordt het ontwerp van de Autopilot en Virtual Testbed verder uitgediept (sectie \ref{sec:Ontwerp}). Vervolgens wordt er ingegaan op de gebruikte algoritmen (sectie \ref{sec:Algoritmes}) en wordt de opbouw van onze software verduidelijkt (sectie \ref{sec:Software}). Ook wijden we uit over de toepassingen van de \textit{GUI} (sectie \ref{sec:GUI}). Tenslotte wordt er afgesloten met de uitgevoerde testen te bespreken (sectie \ref{sec:Testen}) en een kort besluit (sectie \ref{sec:Besluit}). \\