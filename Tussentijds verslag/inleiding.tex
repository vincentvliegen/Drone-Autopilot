{\em Auteur: Laura Vranken}\\

% \emph{[In de inleiding schets je de context, probleemstelling en doelstellingen van het project.  Je kunt ook kort aangeven wat er wel en niet in het verslag staat (bij een tussentijds verslag).]} 

Drones zijn de laatste jaren enorm in populariteit aan het winnen en blijven hierdoor ook positief evolueren in ontwerp en functies. Zo worden ze tegenwoordig namelijk gebruikt zowel met het oog op nuttige toepassingen als ook voor ontspanning. De bekendste toepassingen zijn het leger dat informatie kan winnen over vijandelijk gebied zonder levens van piloten te riskeren en Amazon die zijn bestellingen laat leveren. De toekomst brengt echter veel meer voordelen. Enkele voorbeelden zijn veiligheidsinspectie van windturbines of elektriciteitslijnen, luchtsteun bij search and rescue operaties, bewaking, luchtfotografie... Om deze toepassingen te kunnen verwezelijken, is natuurlijk ook een goede aansturing van belang. De drone kan ofwel aangestuurd worden via een persoon op de grond met een controller ofwel autonoom.(bron: https://www.microdrones.com/en/applications/)\\

Wanneer een drone autonoom functioneert, is de aansturing door de Autopilot van levensbelang. Hij moet namelijk bestand zijn tegen alle weersomstandigheden, zoals wind, regen, sneeuw, ijs en hittegolven. Ook de drone rechtstreeks naar zijn doel leiden, hoort tot zijn taken en bovendien ook veilig laten landen. In geval van meerdere doelen, moet de Autopilot ook de optimale weg kiezen om zijn taak zo snel mogelijk uit te voeren. Aangezien een autonome drone zijn informatie haalt uit de beelden die hij produceert, is de kwaliteit van de camera's ook zeer belangrijk. Indien geen goed beeld verkregen wordt, is de relatieve positiebepaling ook veel moeilijker en minder precies te bepalen en hebben andere toepassingen ook hun nut verloren. Bovendien moet de Autopilot ook rekening houden met het verschil in intensiteit van kleuren op de beelden door bijvoorbeeld invloed van schaduw, felle zon, regendruppels op de camera...\\

In dit verslag gaan we verder in op hoe de autonome aansturing van een drone (en meer bepaald een quadcopter) gebeurt. We gaan uit van een drone waarop twee camera's vast bevestigd zijn op een zekere afstand van elkaar. Beiden zijn ze voorwaarts gericht. Op basis van deze beelden moeten afstand en positie van het doel ingeschat worden en nieuwe bewegingsopdrachten voor de drone gegenereerd worden. Aangezien er geen drones ter beschikking waren, wordt een Virtual Testbed simulator ontworpen. Dit is een softwaresysteem dat een fysieke opstelling met een drone en camera's simuleert. (bron: opgave peno) De simulator gebruikt de bewegingsopdrachten van de Autopilot om deze om te zetten naar verplaatsingen van de drone in de simulator en stuurt de beelden van zijn nieuwe positie weer door naar de Autopilot. \\

De Autopilot en Virtual Testbed moeten zo ontworpen worden dat de drone in staat is om zijn doel, een rode bol, te lokaliseren en er naar toe te vliegen. Dit eventueel onder lichte invloed van wind in willekeurige richtingen. Bovendien moet ook voor beiden een GUI ontworpen worden die ter beschikking staat van de gebruiker. De GUI toont de vooruitgaan en laat de gebruiker toe allerlei informatie op te vragen. \\

De tekst is als volgt opgebouwd. Eerst wordt het ontwerp van de Autopilot en Virtual Testbed verder uitgediept. Vervolgens wordt er ingegaan op de gebruikte algoritmen (sectie 3) en wordt de opbouw van onze software verduidelijkt (sectie 4). Ook wijden we uit over de toepassingen van de GUI in sectie 5. Tenslotte sluiten we af met de uitgevoerde testen (sectie 6) en het besluit. \\