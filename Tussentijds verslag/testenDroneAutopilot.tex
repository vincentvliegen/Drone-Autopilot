\noindent {\em Auteurs: Vincent Vliegen}
\\
\\
De Autopilot is voorzien van enkele \textit{JUnit} testklassen. Deze testen de verschillende gebruikte methodes op hun nauwkeurigheid en correctheid. Zo is het eenvoudiger de capaciteiten van de Autopilot aan te tonen en de beperkingen duidelijk af te bakenen.
\\
\\
De \textit{ImageCalculationsTest} is uitgerust met testen voor elke methode in de \textit{ImageCalculations} klasse. Er is gebruik gemaakt van een anonieme klasse die de \textit{Camera} interface implementeert, zodat er afbeeldingen naar keuze gegenereerd kunnen worden. Deze afbeeldingen zijn omwille van hun grote hoeveelheid informatie beperkt tot een minimale grootte van \(9\text{x}9\) en \(10\text{x}10\) pixels, dit vereenvoudigt immers het testen. Desalniettemin zijn er ook twee .jpg bestanden ter beschikking met elk een representatief beeld van een rode bol. Dit geeft een voldoende nauwkeurige, cirkelvormige afbeelding van een rode bol voor de berekening van desbetreffend middelpunt. De testen tonen aan dat wanneer het centrum van de bol zich buiten beeld bevindt, de benadering van het middelpunt beter is bij berekeningen via het least square algoritme dan bij berekeningen van het zwaartepunt van de zichtbare pixels.%nu nog maar 1 test, onvoldoende om een tabel van resultaten op te stellen, dit is een veronderstelling 
\\
De \textit{PhysicsCalculationsTest} verschaft testmethodes voor de \textit{PhysicsCalculations} klasse. Ook hier zijn anonieme klassen ge\"implementeerd, namelijk voor de \textit{Camera} en \textit{Drone} interfaces. Deze testen geven weer dat de gebruikte formules in iedere situatie voldoende zijn om alle fysische data nauwkeurig te berekenen.


% Geef de resultaten op compacte en heldere manier weer, bv. met tabellen of grafieken. Denk na over de beste manier om de resultaten weer te geven. Ze moeten je conclusies ondersteunen. Formuleer de conclusies.