{\em Auteur: Laura Vranken; redactie: Arne Vlietinck}\\

De Drone Autopilot zorgt voor de besturing van de drone. Het bepaalt zijn positie relatief ten opzichte van zijn doel aan de hand van twee beelden. Deze beelden zijn gemaakt door twee camera's die op de drone gemonteerd staan op een zekere afstand van elkaar. Bovendien kunnen ze opgevraagd worden vanuit de Virtual Testbed. Vervolgens zorgt de Autopilot ervoor dat de drone juist ge\"ori\"enteerd naar het doel staat en er naar toe vliegt. Wanneer de drone zijn doel,  een rode bol, bereikt, moet hij daarin blijven zweven. Tenslotte moet de Autopilot ook rekening houden met een mogelijke invloed van wind die de drone van zijn koers doet afwijken.\\

Ten eerste moeten dus de beelden die de Autopilot van de Virtual Testbed binnen krijgt, geanalyseerd worden. Dit gebeurt door iteratief de integer waarden van elke pixel te vergelijken met de waarde van de kleur rood. Alle rode pixels worden bijgehouden door hun positie ten opzichte van het beeld, uitgedrukt in rij en kolom, op te slagen. We baseren onze komende berekeningen op het midden van de bol. Dat midden kan bepaald worden door het zwaartepunt van de rode pixels te berekenen via het gemiddelde van de opgeslagen co\"ordinaten.\\

Indien de Autopilot geen rode pixels detecteert, zal de drone 360 graden ronddraaien of met andere woorden een yaw beweging uitvoeren, totdat in beide beelden rode pixels verschijnen. In het geval dat de Autopilot slechts rode pixels in een van de twee beelden opmerkt, zal het zichzelf door middel van een yaw beweging in de juiste richting bijsturen. Wanneer uiteindelijk in beide schermen rode pixels gevonden zijn, zal de Autopilot stoppen met draaien en zijn positie tegenover het doel berekenen.\\

Om zijn positie tegenover het doel te berekenen, bepalen we eerste de diepte. Dit kan met behulp van de formule van stereo vision (zie bron) uitgewerkt worden (en zie weergave in bijlage). Vervolgens bepalen we de hoek waaronder de drone een yaw beweging moet uitvoeren om recht naar het doel gericht te zijn. Deze formule kon afgeleid worden uit de goniometrie en wordt weergegeven in bijlage.. Om tenslotte naar het doel te kunnen vliegen, moet een evenwicht gevonden worden tussen pitch en thrust. De pitch hoek wordt gekozen zodat het zwaartepunt van het doel nog juist in beeld blijft. Die hoek is gelijk aan de helft van de verticale hoek die het beeld overspant min de verticale hoek waaronder de bol zich tegenover de drone bevindt (zie bijlage). Wanneer de pitch hoek vastligt, kan de hoeveelheid thrust berekend worden zodat de drone in rechte lijn naar het doel kan vliegen. (uitwerking formule)\\

Tenslotte moet dit proces herhaaldelijk worden uitgevoerd ten gevolge van de invloed van wind. De wind kan de drone namelijk uit koers brengen. Hierdoor zal de drone telkens zijn positie moeten herbepalen en zich opnieuw moeten herori\"enteren. Ook kan de wind ervoor zorgen dat de drone een roll uitvoert. Deze moet eerst gecompenseerd worden, vooraleer we verder onze berekeningen kunnen uitvoeren.\\

De drone bereikt zijn doel wanneer de Autopilot niks anders dan rode pixels opvangt. De drone zal dan de opdracht krijgen om zijn pitch te compenseren en vervolgens enkel via thrust de zwaartekracht tegen te werken. \\

Het effectief laten vliegen van de drone gebeurt in de Virtual Testbed waar de motoren worden aangestuurd. De Autopilot zendt enkel de verhouding in graden per seconden door waaronder pitch, yaw en roll moeten worden uitgevoerd en thrust in Newton. Het is slechts door herhaaldelijk te controleren hoe ver nog gedraaid moet worden, dat kan besloten worden wanneer de beweging volledig uitgevoerd is en wanneer gestopt mag worden.\\

Tegen de volgende deadline is het de bedoeling om de beeldverwerking via OpenCV uit te voeren. Deze bibliotheek kan op een betere en gemakkelijkere manier vormen en kleuren herkennen. Dit speelt in ons voordeel wanneer er meerdere doelen met verschillende vormen en kleuren in beeld zijn. Hierdoor vermijden we het iteratief zoeken en het moeten opslaan van pixels. Aangezien OpenCV op een bepaalde range in RGB waarden kan zoeken, kan ook het probleem van lichtinval en schaduw opgelost worden.
Ook wordt er gezocht naar een betere manier om de roll hoek in de berekeningen te trekken. Zo winnen we tijd aangezien die niet altijd eerst moet gecompenseerd worden voordat de berekeningen kunnen worden uitgevoerd. 