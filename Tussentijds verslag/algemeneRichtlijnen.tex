\em Enkele algemene richtlijnen : 
\begin{itemize}
\item Maak in het hele verslag gebruik van genummerde referenties, zoals hier ge\"\i llustreerd \cite{website:wikibooks-biblio}.  
\item Geef op het niveau van secties en/of subsecties aan wie de auteurs van dat deel zijn.  Een auteur is iemand die de tekst inhoudelijk en vormelijk mee bepaald heeft.    Als de secretaris of iemand anders de tekst vormelijk gewijzigd heeft, bv. om hem meer in lijn met de rest van het verslag te brengen, maar inhoudelijk niets toegevoegd heeft, is die persoon geen auteur maar een redacteur.  Je kan na de auteursnamen eventueel ``redactie: naam'' schrijven om aan te geven dat er substanti\"ele redactie gebeurd is (meer dan pakweg een paar tikfouten verbeteren).  Wanneer alle teamleden samen verantwoordelijk zijn voor een stuk (bv. samenvatting, conclusies) kan je de namen eventueel weglaten.
\item We verwachten dat alle groepsleden een substanti\"ele bijdrage leveren aan het  verslag, en dat dit ook zichtbaar is in de tekst!
\end{itemize}