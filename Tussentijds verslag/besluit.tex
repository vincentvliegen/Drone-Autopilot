\noindent {\em Auteurs: Arne Vlietinck; redactie: Arne Vlietinck}\\
\\
Het doel van dit verslag was een drone met twee camera's naar een rode bol te laten vliegen in een simulatie. Om dit te laten werken, moest de simulator zorgen voor de juiste omgeving, het uitvoeren van de opgegeven draaibewegingen van de drone en dronecamera's en moet een eventuele windkracht ge\"implementeerd worden. 
Bovendien moeten er ook voor de Autopilot veel keuzes gemaakt worden, namelijk over de verwerking van het beeld, het bepalen van de relatieve positie van de drone en het bijsturen d.m.v. rates.
Het is belangrijk om een duidelijk schema voor ogen te houden zodanig dat de uitbreiding voor volgende milestones vlot kan verlopen.




%De programmatie van de Autopilot en simulator bracht verschillende moeilijkheden en doordachte keuzes met zich mee. Deze worden gestructureerd weergegeven in bovenstaand verslag. Daarnaast wordt er een duidelijk schema voor ogen gehouden zodanig de uitbreiding voor volgende milestones vlot kan verlopen.
%Beide programma's voldoen aan de de eerste milestones voor de tussentijdse demo.\\
%Ten slotte wordt er voor de documentatie van volgende milestones verwezen naar het eindverslag.