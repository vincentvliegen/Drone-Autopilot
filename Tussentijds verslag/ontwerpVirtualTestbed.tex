{\em Auteur: Bram Vandendriessche}\\

%TODO!!

%TODO verwijzingen naar Blender, jmonkey en opengl toevoegen
\noindent
Voor de simulator is de belangrijkste keuze uiteraard welke library het meest geschikt is voor het bouwen, weergeven en aanpassen van 3D-werelden. Hiervoor werden Blender\footnote{\url{https://www.blender.org/}}, JMonkeyEngine\footnote{\url{http://jmonkeyengine.org/}} en OpenGL\footnote{\url{https://www.opengl.org/}} onder de loep genomen. 
\\
Blender is een erg uitgebreid programma met tal van mogelijkheden om 3D-objecten en -werelden te maken en manipuleren. Het maakt gebruik van Python, een programmeertaal met een vrij eenvoudige leercurve voor wie al programmeerervaring heeft. Daarentegen is Blender zelf toch uitdagend en tijdrovend om onder de knie te krijgen. Aangezien we echter gepland hadden in Java te werken, moest gezocht worden naar een manier om Java en Python te verbinden. Bovendien zou Blender dan vanuit Java gestart moeten worden, wat een omslachtig proces bleek. Vooral omwille van de extra leertijd en deze laatste eigenschap werd niet voor Blender gekozen. \\
De tweede optie, JMonkeyEngine, leek erg gebruiksvriendelijk, had een goede tutorial en enkele handige ingebouwde functies, zoals het vastpinnen van een camera op een object. Helaas heeft deze \textit{API} een erg beperkte community waardoor problemen vaak zonder hulp van het internet moeten worden opgelost. 
\\
Ondanks de moeilijke beginfase bij het leren van OpenGL werd toch hiervoor geopteerd. OpenGL kan rechtstreeks in Java worden gebruikt, heeft een tal van mogelijkheden, een brede community en er zijn heel wat tutorials beschikbaar. 

%TODO eventueel nog zeggen dat drone weergegeven wordt door balk, kan later nog aangepast worden

%Andere ontwerpkeuzes?