{\em Auteur: Bram Vandendriessche}\\

%TODO!!

%TODO verwijzingen naar Blender, jmonkey en opengl toevoegen
Voor de simulator is de belangrijkste keuze uiteraard welke library het meest geschikt is voor het bouwen, weergeven en aanpassen van 3D-werelden. Hiervoor werden Blender, JMonkeyEngine en OpenGL onder de loep genomen. Blender is een erg uitgebreid programma met tal van mogelijkheden om 3D-objecten en -werelden te maken en manipuleren. Blender leren kost echter tijd wat tijd. Het maakt gebruik van Python, een programmeertaal met vrij eenvoudige leercurve voor wie al programmeerervaring heeft. Omdat we echter gepland hadden in Java te werken, moest gezocht worden naar een manier om Java en Python te verbinden. Bovendien zou Blender dan vanuit Java gestart moeten worden, wat een omslachting proces bleek. Vooral omwille van de leertijd en deze laatste eigenschap werd niet voor Blender gekozen. De tweede optie, JMonkeyEngine, leek erg gebruiksvriendelijk, met een goede tutorial en enkele handige ingebouwde functies, zoals het vastpinnen van een camera op een object. Helaas heeft deze API een erg beperkte community waardoor problemen vaak zonder hulp van internet moeten worden opgelost. Ondanks de moeilijke beginfase bij het leren van OpenGL werd hievoor gekozen. OpenGL kan rechtstreeks in Java worden gebruikt, heeft een tal van mogelijkheden en een brede community en er zijn heel wat tutorials beschikbaar. 

%TODO eventueel nog zeggen dat drone weergegeven wordt door balk, kan later nog aangepast worden

%Andere ontwerpkeuzes?