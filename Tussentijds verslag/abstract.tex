\noindent
{\em Auteur:Laura Vranken }\\
Dit verslag behandelt het ontwerp en de implementatie van een Autopilot en een Virtual Testbed voor een drone. In de simulatie is voor de eerste mijlpaal een rode bol in een witte ruimte te zien en bovendien ook een blauwe drone. Voor de tweede mijlpaal wordt deze wereld uitgebreid met een windkracht in een willekeurige richting. 
Om de drone naar de bol te laten vliegen, is een goede aansturing van belang. Dit is de taak van de Autopilot. Hij bepaalt de vliegroute op basis van informatie, verkregen van twee camera's die zich op de drone bevinden. Hierbij is de relatieve plaatsbepaling van de bol ten opzichte van de drone van belang. De Autopilot leidt daaruit de beweging van de drone af en laat de simulatie deze uitvoeren.\\
De simulator wordt interactief gemaakt door het gebruik van een GUI. Deze geeft de mogelijkheid het camerastandpunt te kiezen en de voltooiingsgraad, de snelheid en de positie van de drone te volgen.


 

% {\em Vat kort samen waar dit verslag over gaat, en wat er in de tekst beschreven staat.}\\