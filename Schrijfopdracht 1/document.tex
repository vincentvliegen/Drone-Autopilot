%%%%%%%%%%%%%%%%%%%%%%%%%%%%%%
%%                                                     %%
%%   LaTeX template voor P&O: Computerwetenschappen.   %%
%%                                                     %%
%%   Schrijfopdracht 1                                 %%
%%                                                     %%
%%   7 oktober 2013                                    %%
%%   Versie 1.1                                        %%
%%                                                     %%
%%%%%%%%%%%%%%%%%%%%%%%%%%%%%%

\documentclass{peno-opdracht1}
\usepackage{comment}

\team{Zilver} % teamkleur

\begin{document}

\maketitle

Ons team bestaat uit Bram Vandendriessche, Matthias Van der Heyden, Jef Versyck, Vincent Vliegen Arne Vlietinck en Laura Vranken. Bram werd aangeduid als CEO en Arne zal de taak van CAO op zich nemen. Beide proberen de vergaderingen en verslaggeving ordentelijk te laten verlopen, daarnaast houden ze ook rekening met de milestones. \\
Om op een effici\"ente manier het geheel te kunnen realiseren, wordt het team in twee verdeeld. De ene groep (Bram, jef en Arne) buigt zich over de simulator. Het andere team  (Matthias, Vincent en Laura) zorgt voor de autopilot. Steeds is er ook een iemand per groep die verantwoordelijk is voor de connectie tussen de twee programma's. Dit laat toe dat er een vlotte communicatie is tussen beide groepen.
Een gedetailleerdere taakverdeling kan teruggevonden worden in de bijlage.


\begin{comment}
	Na de eerste volledige sessie wordt max. 2 pagina's verwacht. Beschrijf op een zeer hoog
	niveau wat je gaat realiseren, hoe je ontwerp er verwacht wordt uit te zien, een ruwe schets
	van de planning en geef de verantwoordelijkheden die daar aan gekoppeld zijn weer. Al
	het belangrijke van jullie eerste brainstormsessie zou hierin vermeld moeten zijn. De tekst
	wordt geacht op een bladzijde te passen, en bijhorend wordt er een schets/tabel/figuur
	verwacht die duiding geeft bij een bepaald aspect uit deze tekst.
	
	\emph{Deadline:11/10/2016 voor 24u. Insturen per mail naar begeleider(s)}
	\newline
	\newline
	
	Brainstorming: 
	\begin{enumerate}
	
	\item Voorstellen drone: balk om de botsingen te dekken. Massamiddelpunt centrum. Hoogte drone defini\"eren (fysische eigenschappen verder uitwerken). 
	\item Simulator: grondopp, display wit, rode bol, zwaartekracht dus hierdoor kennis vn de grondopp.
	\newline
	\item Dieptezicht
	\item Gekregen Javacode analyseren
	\item Punt van de twee camera's vastleggen. Hieruit volgen dan de berekenen.
	\item RPY-hoeken 
	\item Hoe pixels omzetten nr beeld? Veel van te vinden online.
	\item Eigenschappen van de drone? (vrij te kiezen?)
	\item Hoe afgaan op de pixels van de bol? 
	
	\end{enumerate}
\end{comment}
	








\end{document}
