% \emph{[In de inleiding schets je de context, probleemstelling en doelstellingen van het project.  Je kunt ook kort aangeven wat er wel en niet in het verslag staat (bij een tussentijds verslag).]} 
{\em Auteur: Jef Versyck}
\\\\
Voor het vak P\&O Computerwetenschappen wordt er gevraagd een simulatie te maken van een drone en hiervoor een autopiloot te ontwerpen. Beiden moeten autonoom kunnen werken. De simulatie moet een correcte weergave van de gevraagde voorwaarden weergeven, terwijl de autopiloot vooropgestelde mijlpalen moet voltooien. De bedoeling van het project is om verschillende technische maar ook sociale vaardigheden aan te scherpen en het project tot een goed einde te brengen.
Voor het verder verloop van deze tekst zal de simulatie benoemd worden als Testbed en de autopiloot als Autopilot.
\\
\\
\noindent
Dit semester zijn er vier mijlpalen meegegeven. De eerste vereist dat de drone alle \textit{Target} polyhedra raakt, zonder \textit{Obstacle} polyhedra te raken. Een polyhedron (mv. polyhedra) is een 3D-figuur die in dit geval, volgens de opdracht, enkel bestaat uit driehoeken. Ondertussen zijn er ook onbekende windtranslaties en -rotaties die inwerken op de drone. Om de mijlpaal te vergemakkelijken, staan alle objecten drie drone-diameters uit elkaar en zijn ze zichtbaar vanuit de startpositie. \\
Voor de tweede mijlpaal moet een verbinding gemaakt kunnen worden tussen het voorziene testbed en de eigen Autopilot.\\
De derde mijlpaal stelt dat de drone in de wereld rond moet vliegen en elk object van de wereld moet scannen. Deze objecten moeten dan in een apart venster opnieuw gecre\"eerd worden terwijl de drone ze aan het scannen is. \\
De vierde en laatste mijlpaal is identiek aan de eerste mijlpaal met als enige verandering dat de afstanden en de zichtbaarheid van alle objecten nu niet meer gelden. Dit kan er voor zorgen dat een \textit{Obstacle} polyhedron zich in een \textit{Target} polyhedron bevindt. Hier moet dus rekening mee gehouden worden.
\noindent
\\
\\
De tekst is als volgt opgebouwd:\\
Eerst is er een sectie over de veranderingen van het Testbed ten opzichte van vorig semester. Dit wordt gevolgd door een bespreking van de gebruikte vliegstrategie van de Autopilot en de moeilijkheden die daarbij ondervonden zijn. 
Daarop volgt het deel over de beeldherkenning en de grafische voorstelling van de gescande objecten voor de Autopilot. Dit onderdeel hoort bij de Autopilot, maar is als een afzonderlijk deel neergeschreven omwille van de omvang. Het deel erna behandelt de testen en de huidige resultaten ten opzichte van de gevraagde mijlpalen. Ten slotte is er het besluit.
%Eerst wordt de beeldverwerking en de grafische voorstelling van de Autopilot verder uitgediept (sectie \ref{sec:Beeldverwerking}). Vervolgens wordt er ingegaan op de gebruikte vliegstrategie (sectie \ref{sec:Vliegstrategie}) en worden de mogelijkheden van het Virtual Testbed verduidelijkt (sectie \ref{sec:Virtual Testbed}). Tot slot worden de uitgevoerde testen besproken (sectie \ref{sec:Testen}) en wordt dit aangevuld met een kort besluit. 
